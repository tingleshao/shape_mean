% ----------------------------------------------------------------
% Article Class (This is a LaTeX2e document)  ********************
% ----------------------------------------------------------------
\documentclass[12pt]{article}
\usepackage[english]{babel}
\usepackage{amsmath,amsthm}
\usepackage{amsfonts}
\usepackage{hyperref}
% THEOREMS -------------------------------------------------------
\newtheorem{thm}{Theorem}[section]
\newtheorem{cor}[thm]{Corollary}
\newtheorem{lem}[thm]{Lemma}
\newtheorem{prop}[thm]{Proposition}
\theoremstyle{definition}
\newtheorem{defn}[thm]{Definition}
\theoremstyle{remark}
\newtheorem{rem}[thm]{Remark}
\usepackage{amssymb}
\usepackage{setspace}
\numberwithin{equation}{section}
\doublespacing
% ----------------------------------------------------------------
\begin{document}


\title{A Study of Shape Means}%
\author{Chong Shao}%
%\address{}%
%\thanks{}%
%\date{}%
% ----------------------------------------------------------------

\maketitle
% ----------------------------------------------------------------
\section{Introduction}
The first part of this paper discusses Stephen Huckemann's paper ``On the meaning of mean shape" discusses about three kinds of means: Intrinsic mean, extrinsic mean and residue mean. Each has a corresponding quotient mean: Intrinsic, Ziezold and Procrustean mean, respectively. \\[0.2cm]
One important property that is presented in this paper is the stability of the three means: intrinsic and Ziezold means are stable while Procrustes means are generally not. Here stability means the little perturbation in samples would not make great changes in the population's sample mean. \\[0.2cm]
The second part of this paper is a summation of Ellingson et. al.'s paper ``Nonparametric Estimation for Extrinsic Mean Shapes of Planar Contours". This paper describes one approach in computing mean for a population of nonparametric representation of 2D planar contours.
\section{Definition of means}
\section{Proof of stability}
\section{1:3 Property}
\section{Extrinsic mean shapes of planar contours}
\section{Veronese-Whitney embedding}
1. If $\mathbf{V}$ is a vector space over the commutative field $\mathbb{F}$, then the set of all one dimensional linear subspaces of $\mathbf{V}$ is the \emph{projective space} of $\mathbf{V}$, and is labeled $\mathbf{P(V)}$. If $\mathbf{V} = \mathbb{F}^{d+1}$, the standard notation $P(\mathbb{F}^{d+1}) = \mathbb{F}P^d$ is used. \\[0.2cm]
2. Let $P(\mathbf{H})$ be the \emph{projective space} corresponding to the Hilbert space $\mathbf{H}$.\\[0.2cm]
3. We embed $P(\mathbf{H})$ is $\mathcal{L}_{HS} = \mathbf{H} \otimes \mathbf{H}$, the space of Hilbert-Schmidt operators of $\mathbf{H}$ into itself, via the Veronese-Whitney embedding $j$ given by 
\[j([\gamma]) = \frac{1}{||\gamma||^2} \gamma \otimes \gamma\]
If $||\gamma|| = 1$, this definition can be reformulated as 
\[j([\gamma]) = \gamma \otimes \gamma\]
Given that $\gamma^*(\beta) = <\beta , \gamma>$, equation above is equivalent to
\[j([\gamma])(\beta)= <\beta , \gamma> \gamma\]
4?. The \emph{range of this embedding} is the submanifold $\mathcal{M}_1$ of rank one Hilbert-Schmidt operators of $\mathbf{H}$.  $P(\mathbf{H})$ is a Hilbert manifold which is embedded in the Hilbert space $\mathcal{L}_{HS}$, for any probability measure $Q$ on a Hilbert manifold $M$ embedded in a Hilbert space, the extrinsic mean of Q w.r.t. that embedding can be defined. \\[0.2cm]
5. \textbf{Def. of mean} \\[0.2cm]
Assume $\Gamma$ is a random object in $P(\mathbf{H})$, with $||\Gamma||^2$ finite. The Whitney -Veronese extrinsic mean of $\Gamma$ exists if and only if $E(\frac{1}{||\Gamma||^2}\Gamma \otimes \Gamma)$ has a simple largest eigenvalue, and in this case the extrinsic mean is $\mu_E = [\gamma]$, where $\gamma$ is an eigenvector for this eigenvalue.\\[0.2cm]
6. If $\gamma_1 \dots \gamma_n$ is a random sample of size $n$ from $\Gamma$, then the extrinsic sample mean $\hat{\mu}_{E,n}$ is the projective point of the eigenvector corresponding to the largest eigenvalue of $\frac{1}{n}\sum_{i=1}^{n}\frac{1}{||\gamma_i||^2}\gamma_i \otimes \gamma_i$ \\[0.2cm]
7. In order to take the contour shape into consideration, the distance between two parameterized contours are defined as:
\[d(\gamma_1, \gamma_2) = \text{sup}_s|\kappa_1(s)-\kappa_2(s)|+\text{sup}_s|\gamma_1(s)-\gamma_2(s)|\] 
$\kappa_1$ and $\kappa_2$ are the curvatures of $\gamma_1$ and $\gamma_2$ respectively. \\[0.2cm]
8. Finite representation of planar contours\\[0.2cm]
We now switch to the finite representation of planar contours where a contour is represented by $k$ landmarks uniformly sampled. The starting point is chosen to be the point furthest from the center of gravity of that contour. \\[0.2cm]
9. Therefore we need a discrete version of VW embedding. \\[0.2cm]

\section{Experiments}


\section{References}
[1] Stephen Huckemann, ``On the Meaning of Mean Shape" \\[0.2cm]
[2] Contour shape mean paper\\[0.2cm]
[3] Hilbert Schmidt operator \href{http://en.wikipedia.org/wiki/Hilbert\%E2\%80\%93Schmidt_operator}{link}\\[0.2cm]
[4] Veronese surface \href{http://en.wikipedia.org/wiki/Veronese_surface}{link}\\[0.2cm]
[5] Whitney embedding theorem \href{http://en.wikipedia.org/wiki/Whitney_embedding_theorem}{link}\\[0.2cm]
[6] The Statistics of shapes book on Steve's shelf\\[0.2cm]
[7] Contour shape mean paper 2013 version
\end{document}
% ----------------------------------------------------------------
